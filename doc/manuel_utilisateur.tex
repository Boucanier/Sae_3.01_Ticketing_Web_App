\documentclass[12pt, a4paper]{article}
\usepackage{geometry}
\geometry{margin=2.5cm}
\usepackage{graphicx}
\usepackage[utf8]{inputenc}
\usepackage[french]{babel}

\title{Manuel d'utilisation de \textbf{Ticket App}}
\author{Jules CHIRON, Matis RODIER, Thomas GODINEAU | INF2 FI A}

\usepackage[T1]{fontenc}
\begin{document}
\maketitle

\begin{figure}[h]
    \includegraphics[width=0.6\textwidth]{annexes/logo_uvsq}
\end{figure}

\tableofcontents{}

\section*{Introduction}
\addcontentsline{toc}{section}{Introduction}

\textbf{Ticket App} est une application web de ticketing en PHP.\@
Ce document a pour but de vous guider dans l'utilisation de notre application.

\bigskip
\noindent Il existe plusieurs types d'utilisateurs sur notre plateforme:
\begin{itemize}
    \item Les utilisateurs
    \item Les techniciens
    \item L'administrateur web
    \item L'administrateur système
\end{itemize}

\bigskip
\noindent Ce manuel contient une section pour chaque type d'utilisateur.

\section{Utilisateur}

Chaque utilisateur peut créer un ticket et consulter l'avancée de ses tickets.

\subsection*{Création de compte et connexion}

Pour vous connecter ou créer un compte depuis la page d'accueil, cliquez sur le bouton \textit{\textbf{Se connecter}} en haut à droite de la page.

\noindent Une fois sur la page de connexion, vous pouvez soit \textbf{créer un compte} depuis l'encadré de gauche,
soit \textbf{vous connecter} depuis l'encadré de droite.

Pour créer un compte, il faut que vous renseignez: votre \textit{nom}, votre \textit{prénom}, un \textit{login} de votre choix et un \textit{mot de passe}. 
il faut ensuite remplir le \textit{captcha} (simple calcul) et cliquer sur le bouton \textit{\textbf{Créer}}.
Vous serrez alors automatiquement connecté.

\bigskip
Pour vous connecter, il faut que vous renseignez votre \textit{login} et votre \textit{mot de passe} et cliquer sur le bouton \textit{\textbf{Se connecter}}.
Vous serrez alors automatiquement redirigé vers votre tableau de bord.

\subsection*{Tableau de bord}

Le tableau de bord d'un utilisateur permet de \textbf{créer des tickets} et de \textbf{consulter} l'avancée de ses tickets.

\subsubsection*{Création de ticket}

Pour créer un ticket, cliquez sur le bouton \textit{\textbf{Créer un ticket}} depuis votre tableau de bord.
Sur la page de création de ticket, il faut renseigner un \textit{libellé} pour le ticket.
Ce \textit{libellé} doit permettre de reconnaître rapidement votre problème (par exemple: \textit{Prise cassée}, \textit{Problème de boot}, \ldots)
Il faut ensuite indiquer la \textit{salle} de votre problème.
Seules les salles du département informatique sont disponibles.
S'il s'agit d'un problème dans une autre salle ou sur un ordinateur personnel, veuillez sélectionner le champ \textit{\textbf{Autre}}.
Vous devez également indiquer le \textit{niveau d'urgence} de votre problème.
Ce niveau est compris \textbf{entre 1 et 4}, 1 étant un problème \textbf{peu important} et 4 un problème \textbf{très urgent}.
Enfin, décrivez précisément votre problème dans le champ \textit{Description}.

\bigskip
\noindent Une fois ces informations renseignées, cliquez sur le bouton \textit{\textbf{Valider}} pour valider la création de votre ticket.

\subsubsection*{Consulter un ticket}

Vous pouvez consulter les détauls de tous les tickets que vous avez créé depuis votre tableau de bord.
Pour cela, cliquez sur le bouton \textit{\textbf{Détails}} à côté du ticket que vous souhaitez consulter.
Vous retrouverez alors toutes les informations que vous avez renseignées lors de la création du ticket ainsi que son état actuel.

\section{Technicien}

Chauqe technicien peut \textbf{prendre en charge} des tickets. Il peut également \textbf{clore les tickets} qu'il a pris en charge.

\subsection*{Connexion}

Pour vous connecter depuis la page d'accueil, cliquez sur le bouton \textit{\textbf{Se connecter}} en haut à droite de la page.
Renseignez ensuite le \textit{login} et le \textit{mot de passe} qui vous ont été fournis et cliquez sur le bouton \textit{\textbf{Se connecter}}.
Vous serrez alors automatiquement redirigé vers votre tableau de bord.

\subsection*{Prise en charge de ticket}

Pour afficher la liste des tickets non affectés, cliquez sur le bouton \textit{\textbf{Tickets disponibles}} sur les boutons de navigation du site.
Pour prendre en charge un ticket, cliquez sur le bouton \textit{\textbf{Prendre}} à côté du ticket que vous souhaitez prendre en charge.
Les détails du ticket s'afficheront alors et vous pourrez cliquer sur le bouton \textit{\textbf{Prendre en charge}} pour confirmer la prise en charge du ticket ou sur le bouton \textit{\textbf{Retour}} pour annuler.

\bigskip
\noindent Tous les tickets que vous avez pris en charge sont affichés sur votre tableau de bord.

\subsection*{Fermeture de ticket}

Pour pouvoir clore un ticket, il faut vous rendre sur votre tableau de bord.
Cliquez ensuite sur le bouton \textit{\textbf{Détails}} à côté du ticket que vous souhaitez clore.
Les détails du ticket s'afficheront alors et vous pourrez cliquer sur le bouton \textit{\textbf{Clore}} pour clore le ticket ou sur le bouton \textit{\textbf{Retour}} pour annuler.

\section{Administrateur web}

L'administrateur web peut \textbf{gérer les techniciens}, \textbf{gérer les utilisateurs} et \textbf{gérer les tickets} de la plateforme.

\subsection*{Connexion}

Pour vous connecter depuis la page d'accueil, cliquez sur le bouton \textit{\textbf{Se connecter}} en haut à droite de la page.
Renseignez ensuite le \textit{login} et le \textit{mot de passe} qui vous ont été fournis et cliquez sur le bouton \textit{\textbf{Se connecter}}.
Vous serrez alors automatiquement redirigé vers votre tableau de bord.

\subsection*{Gestion des techniciens}

Pour afficher la page de gestion des techniciens, cliquez sur le bouton \textit{\textbf{Gestion des techniciens}} sur les boutons de navigation du site.
Cette page contient la liste de tous les techniciens avec le nombre de tickets dont ils ont \textbf{actuellement} la charge.

\bigskip
\noindent Vous pouvez également ajouter des techniciens depuis cette page en utilisant l'encadré de droite ``\textit{\textbf{Ajouter un technicien}}''.
Pour ce faire, vous devez renseigner le \textit{nom}, le \textit{prénom}, le \textit{login} et le \textit{mot de passe} du technicien puis cliquer sur le bouton \textit{\textbf{Créer}}.

\subsection*{Gestion des utilisateurs}

Pour afficher la page de gestion des utilisateurs, cliquez sur le bouton \textit{\textbf{Gestion des utilisateurs}} sur les boutons de navigation du site.
Cette page contient la liste de tous les utilisateurs (\textit{utilisateurs et techniciens}).
Vous avez la possibilté de supprimer un compte en cliquant sur le bouton \textit{\textbf{Supprimer le compte}} à côté du compte que vous souhaitez supprimer.
Vous pouvez également modifier le mot de passe d'un utilisateur en rentrant un nouveau mot de passe et en cliquant sur le bouton \textit{\textbf{Modifier le mot de passe}} sur la ligne de l'utilisateur.

\subsection*{Gestion des tickets}

L'ensemble des tickets \textbf{non clos} se trouve sur le \textbf{tableau de bord} de l'administrateur web.
Pour modifier un ticket, il faut cliquer sur le bouton \textit{\textbf{Détails}} à côté du ticket que vous souhaitez modifier.

\subsubsection*{Modification de ticket}

Sur la page de détails d'un ticket, vous pouvez modifier son \textit{libellé} et son \textit{niveau d'urgence}.
Vous pouvez également changer l'état du ticket avec le champ \textit{\textbf{Nouvel état}}.
\textbf{ATTENTION!} en changeant l'état d'un ticket en \textit{clos}, vous ne pourrez plus le modifier par la suite.
De la même manière, vous pouvez affecter le ticket à un technicien dans le champ \textit{\textbf{Affecter un technicien}}.
Le technicien peut être modifié à tout moment.

\bigskip
\noindent Pour valider les modifications, cliquez sur le bouton \textit{\textbf{Modifier}}.

\section{Administrateur système}

\end{document}