\documentclass[12pt, a4paper]{article}
\usepackage{geometry}
\geometry{margin=2.5cm}
\usepackage{graphicx}
\usepackage[utf8]{inputenc}
\usepackage[french]{babel}

\title{Manuel d'utilisation de \textbf{Ticket App}}
\author{Jules CHIRON, Matis RODIER, Thomas GODINEAU | INF2 FI A}

\usepackage[T1]{fontenc}
\begin{document}
\maketitle

\begin{figure}[h]
    \includegraphics[width=0.6\textwidth]{annexes/logo_uvsq}
\end{figure}

\tableofcontents{}

\section*{Introduction}
\addcontentsline{toc}{section}{Introduction}

\textbf{Ticket App} est une application web de ticketing en PHP.\@
Ce document a pour but de vous guider dans l'utilisation de notre application.

\bigskip
\noindent Il existe plusieurs types d'utilisateurs sur notre plateforme:
\begin{itemize}
    \item Les utilisateurs
    \item Les techniciens
    \item L'administrateur web
    \item L'administrateur système
\end{itemize}

\bigskip
\noindent Ce manuel contient une section pour chaque type d'utilisateur.

\section{Utilisateur}

Chaque utilisateur peut créer un ticket et consulter l'avancée de ses tickets.

\subsection*{Création de compte et connexion}

Pour vous connecter ou créer un compte depuis la page d'accueil, cliquez sur le bouton \textit{\textbf{Se connecter}} en haut à droite de la page.

\noindent Une fois sur la page de connexion, vous pouvez soit \textbf{créer un compte} depuis l'encadré de gauche,
soit \textbf{vous connecter} depuis l'encadré de droite.

Pour créer un compte, il faut que vous renseignez: votre \textit{nom}, votre \textit{prénom}, un \textit{login} de votre choix et un \textit{mot de passe}. 
il faut ensuite remplir le \textit{captcha} (simple calcul) et cliquer sur le bouton \textit{\textbf{Créer}}.
Vous serrez alors automatiquement connecté.

\bigskip
Pour vous connecter, il faut que vous renseignez votre \textit{login} et votre \textit{mot de passe} et cliquer sur le bouton \textit{\textbf{Se connecter}}.
Vous serrez alors automatiquement redirigé vers votre tableau de bord.

\subsection*{Tableau de bord}

Le tableau de bord d'un utilisateur permet de \textbf{créer des tickets} et de \textbf{consulter} l'avancée de ses tickets.

\subsubsection*{Création de ticket}

Pour créer un ticket, cliquez sur le bouton \textit{\textbf{Créer un ticket}} depuis votre tableau de bord.
Sur la page de création de ticket, il faut renseigner un \textit{libellé} pour le ticket.
Ce \textit{libellé} doit permettre de reconnaître rapidement votre problème (par exemple: \textit{Prise cassée}, \textit{Problème de boot}, \ldots)
Il faut ensuite indiquer la \textit{salle} de votre problème.
Seules les salles du département informatique sont disponibles.
S'il s'agit d'un problème dans une autre salle ou sur un ordinateur personnel, veuillez sélectionner le champ \textit{\textbf{Autre}}.
Vous devez également indiquer le \textit{niveau d'urgence} de votre problème.
Ce niveau est compris \textbf{entre 1 et 4}, 1 étant un problème \textbf{peu important} et 4 un problème \textbf{très urgent}.
Enfin, décrivez précisément votre problème dans le champ \textit{Description}.

\bigskip
\noindent Une fois ces informations renseignées, cliquez sur le bouton \textit{\textbf{Valider}} pour valider la création de votre ticket.

\subsubsection*{Consulter un ticket}

Vous pouvez consulter les détauls de tous les tickets que vous avez créé depuis votre tableau de bord.
Pour cela, cliquez sur le bouton \textit{\textbf{Détails}} à côté du ticket que vous souhaitez consulter.
Vous retrouverez alors toutes les informations que vous avez renseignées lors de la création du ticket ainsi que son état actuel.

\section{Technicien}

\section{Administrateur web}

\section{Administrateur système}

\end{document}