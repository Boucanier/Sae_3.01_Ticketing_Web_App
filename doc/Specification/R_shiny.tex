\documentclass[12pt, a4paper]{article}
\usepackage{geometry}
\geometry{margin=2.5cm}
\usepackage{graphicx}
\usepackage{amsmath}
\usepackage{enumitem}
\usepackage[utf8]{inputenc}
\usepackage[french]{babel}
\usepackage{listings}
\usepackage{xcolor}
\title{SAÉ 3.01 Rapport de probas/stats}
\author{Jules CHIRON, Matis RODIER, Thomas GODINEAU | INF2 FI A}
\date{12 janvier 2024}

\NewDocumentCommand{\codeword}{v}{%
\texttt{\textcolor{blue}{#1}}%
}

\usepackage[T1]{fontenc}
\begin{document}
\maketitle

\begin{figure}[h]
    \includegraphics[width=0.6\textwidth]{../annexes/logo_uvsq}
\end{figure}

\tableofcontents{}

\section*{Introduction}
\addcontentsline{toc}{section}{Introduction}

\section{Présentation}

\section{Notre page}

Notre page shiny contient 3 principaux éléments:
\begin{itemize}
    \item L'interface utilisateur (\textit{ui})
    \item La partie serveur (fonction) (\textit{server})
    \item ShinyApp (fonction)
\end{itemize}
Le code de notre page ainsi que les fichiers csv utilisés se trouvent dans le dossier fourni avec ce rapport.

\subsection*{Ordre du programme}

Nous commençons notre programme en chargeant la bibliothèque \textbf{shiny} avec la fonction \codeword{library(shiny)}.
Puis on règle le port par défaut de l'application pour qu'elle soit toujours accessible sur le même port (\textit{port 3000})
avec la fonction \codeword{options(shiny.port = 3000)}.
\bigskip

Nous ajoutons ensuite les éléments \textit{ui} et \textit{server}.
Puis nous lançons l'application avec la fonction \codeword{shinyApp(ui, server)}.

\subsection*{Interface utilisateur}

L'élément \textbf{ui} correspond à une \textbf{fluidPage} qui contient l'interface utilisateur de notre application.
Notre \textbf{fluidPage} contient 3 éléments:
\begin{itemize}
    \item Un \textbf{titlePanel}: titre
    \item Deux \textbf{sidebarLayout}: choix des \textit{paramètres} et représentation des \textit{résultats}
\end{itemize}
L'élément \textbf{titlePanel} permet de mettre un titre à notre page, on l'a appelée ici `Statistiques'.
Nos \textbf{sidebarLayout} contiennent chacun un \textbf{sidebarPanel} et un \textbf{mainPanel}.
\bigskip

Le premier \textbf{sidebarLayout} sert à choisir les paramètres pour la première statistique (le pourcentage de tickets selon leur status).
Son \textbf{sidebarPanel} contient un \textbf{titlePanel} permettant de donner un titre au panel, ainsi que deux \textbf{inputs} (\textit{élément de choix}):
\begin{itemize}
    \item Le premier input est un \textit{selectInput} qui permet de choisir entre les différents status de ticket
    (open (\textit{ouvert}), in\_progress (\textit{en cours}), closed (\textit{fermé}))
    \item Le deuxième input est un \textit{numericInput} qui permet de choisir le nombre de tickets à étudier (entre 1 et 40)
\end{itemize}
Son \textbf{mainPanel} contient deux outputs (\textit{sorties}):
\begin{itemize}
    \item Le premier est de type \textit{verbatimTextOutput} > la sortie sera sous forme de texte
    \item Le deuxième est de type \textit{plotOutput} > la sortie sera un graphique
\end{itemize}
\bigskip

Le deuxième \textbf{sidebarLayout} sert aux choix des paramètres de la seconde statistique (le pourcentage de connexions réussies).
Son \textbf{sidebarPanel} contient un \textbf{titlePanel} et un seul input:
\begin{itemize}
    \item L'input est un \textit{numericInput}  qui permet de choisir le nombre de connexions que l'on souhaite étudier.
\end{itemize}
Son \textbf{mainPanel} contient deux \textit{outputs}:
\begin{itemize}
    \item Les deux sont de type \textit{plotOutput} > il y aura donc deux graphiques.
\end{itemize}

\subsection*{Partie serveur}

Notre fonction \textbf{server} commence par récupérer les données des fichiers csv (\textit{tickets.csv} et \textit{connexions.csv}).
Nous avons créés ces deux fichiers en s'inspirant des données présentes dans la base de données de l'application.

\noindent Nous paramétrons ensuite les représentations de nos différents \textit{outputs}.
Ils sont listés dans l'ordre décrit dans la partie précédente:

\begin{itemize}
    \item Le premier est une zone de texte contenant le pourcentage de ticket selon le type choisis dans le \textit{selectInput}.
    \item Le deuxième est un graphique en camembert dont chaque partie représente le pourcentage de chaque status de ticket.
    \item Le troisième est encore un graphique en camembert dont chaque partie représente le pourcentage de connexions réussies ou échouées.
    \item Le quatrième graphique est la représentation du pourcentage de connexions réussies sous forme de points.
\end{itemize}

\section{Déploiement}

\end{document}